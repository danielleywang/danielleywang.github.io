\documentclass{article}

\usepackage{purrsonal}

\DeclareMathOperator{\im}{im}
\DeclareMathOperator{\lcm}{lcm}

\begin{document}
\section{Elementary number theory}

The goal of this lecture is to prove Fermat's little
theorem.

\begin{theorem}
Let $p$ be a prime, and let $a$ be any integer.
Then $a^p - a$ is divisible by $p$.
\end{theorem}

\subsection{Modular arithmetic [2.7, 2.9]}

An \emph{equivalence relation} on a set $S$ is a relation
$\sim$ between certain pairs of elements of $S$.
We write $a \sim b$ if $a$ and $b$ are \emph{equivalent}.
An equivalence relation is required to be
\begin{itemize}
	\item \emph{transitive}: if $a \sim b$ and $b \sim c$,
	then $a \sim c$.

	\item \emph{symmetric}: if $a \sim b$, then $b \sim a$.

	\item \emph{reflexive}: for all $a$, $a \sim a$.
\end{itemize}
An equivalence relation $\sim$ partitions $S$ into 
\emph{equivalence classes}.

\begin{definition}
Let $n$ be a positive integer. For integers
$a, b$, we write
\[
	a \equiv b \pmod n
\]
if $a - b$ is divisible by $n$, i.e., $a - b = nk$
for some integer $k$.
\end{definition}

\begin{lemma}[Addition and multiplication modulo $n$]
If $a' \equiv a \pmod n$ and $b' \equiv b \pmod n$,
then $a' + b' \equiv a + b \pmod n$ and
$a'b' \equiv ab \pmod n$.
\end{lemma}
\begin{proof}
Suppose $a' = a + nk$ and $b' = b + n\ell$.
Then
\[
	a' + b' = (a + b) + n(k + \ell),
\]
and
\[
	a'b' = ab + n(a\ell + bk + k\ell). \qedhere
\]
\end{proof}

\begin{definition}
Let $\ZZ/n\ZZ$ denote the set of equivalence classes
of $\ZZ$ with respect to the equivalence relation
$\equiv$. These equivalence classes are also referred
to as \emph{congruence classes} modulo $n$.
\end{definition}

By the lemma above, addition and multiplication
of congruence classes modulo $n$ is well-defined.
If we write $\ol a$ to denote the congruence class
of $a$, then 
\[
	\ol a + \ol b = \ol{a + b}, 
\]
and similarly 
\[
	\ol a \ol b = \ol{ab}.
\]

The associative, commutative, and distributive laws
carry over for addition and multiplication of elements
of $\ZZ/n\ZZ$.

\begin{example}
$\ZZ/6\ZZ$ has $6$ elements. 
The elements $\ol 2$ and $\ol 8$ are the same element
since $2 \equiv 8 \pmod 6$.

We have $\ol 2 \cdot \ol 5 = \ol{10}$, and
$\ol 8 \cdot \ol 5 = \ol{40}$. Fortunately,
$\ol{10} = \ol{40}$ since $10 \equiv 40 \pmod 6$. We
usually take the remainder when divided by $6$ and
say $\ol 2 \cdot \ol 5 = \ol 4$.
\end{example}

\subsection{Bezout's lemma [2.3]}
We recall division with remainder: let $n$ be an integer,
and let $a$ be a positive integer. Then there
exists an integer $q$ and an integer $0 \le r < a$
such that
\[
	n = aq + r.
\]

\begin{definition}
Let $a$ and $b$ be integers, not both zero. 
The \emph{greatest common
divisor} of $a$ and $b$, denoted $\gcd(a, b)$, is
the largest integer which divides both $a$ and $b$.
If $\gcd(a, b) = 1$, we say that $a$ and $b$ are
\emph{coprime} or \emph{relatively prime}.
\end{definition}

The GCD satisfies the property that
\[
	\gcd(a, b) = \gcd(a + bk, b)
\]
for any integer $k$. Indeed, if $d$ divides both $a$
and $b$, then $d$ divides both $a + bk$ and $b$, and
conversely.

As such, we can compute GCD's using the \emph{Euclidean
algorithm}, which works by repeated division with remainder.

\begin{example}
For example, for $a = 314$, $b = 136$, since
\[
	314 = 2 \cdot 136 + 42, \quad 136 = 3 \cdot 42 + 10,
	\quad 42 = 4\cdot 10 + 2,
\]
we have
\begin{align*}
	\gcd(314, 136)
	= \gcd(42, 136) = \gcd(42, 10)
	= \gcd(2, 10) = 2.
\end{align*}
\end{example}

\begin{proposition}[Bezout's lemma]
For any integers $a$ and $b$, not both zero, there exist integers $r$ and 
$s$ such that
\[
	\gcd(a, b) = ra + sb.
\]
\end{proposition}
\begin{proof}
Let $d = \gcd(a, b)$
Let $\ell$ be the smallest positive integer that
can be expressed as 
\[ 
	\ell = ra + sb 
\]
for some $r$ and $s$.

We claim that $\ell | a$. Use division with remainder to
write
\[
	a = \ell q + m
\]
for $0 \le m < \ell$.
Then $m$ can also be expressed in the form $ra + sb$:
\[
	m = a - \ell q =
	a - q(ra + sb) = (1 - qr)a - (qs)b.
\]
Since $\ell$ was assumed to be minimal, $m = 0$,
so $\ell | a$.

Similarly, $\ell | b$, so $\ell$ divides both $a$ and $b$. 
Since $d$ is the greatest common divisor,
\[
	\ell \le d.
\]

On the other hand, $d$ divides both $ra$ and
$sb$, so $d$ also divides $\ell$, so
\[
	d \le \ell. 
\]
Thus, $\ell = d$.
\end{proof}

\begin{corollary}
Let $e$ be an integer which divides both $a$ and $b$.
Then $e$ divides $\gcd(a, b)$.
\end{corollary}
\begin{proof}
Let
\[
	\gcd(a, b) = ra + sb.
\]	
Since $e$ divides both terms on the right hand side,
it also divides $\gcd(a, b)$.
\end{proof}

\begin{corollary}
\label{prime}
Let $p$ be a prime, and let $a$ and $b$ be integers.
If $p | ab$, then $p | a$ or $p | b$.	
\end{corollary}
\begin{proof}
Suppose that $p$ divides $ab$, but $p$ does not
divide $a$. 

Since $p$ is prime, $\gcd(a, p) = 1$, so by Bezout's lemma
there exist $r, s \in \ZZ$ such that
\[
	1 = ra + sp.
\]
Multiplying both sides by $b$,
\[
	b = rab + spb.
\]

Both terms on the right are multiples of $p$ by the 
assumption $p | ab$, so $p | b$.
\end{proof}

\begin{corollary}[$\ZZ/p\ZZ$ has inverses]
\label{inverse}
Let $p$ be a prime, and
let $a$ be an integer which is not divisible by $p$.
There exists an integer $b$ such that
$ab \equiv 1 \pmod p$.
\end{corollary}
\begin{proof}
	As in the proof above, there exist $r, s \in \ZZ$
	such that
	\[
		1 = ra + sp.
	\]
	So $ra \equiv 1 \pmod p$. Clearly, we can take $b = r$.
\end{proof}

\subsection{Proof of Fermat's little theorem}

\begin{proof}
If $a$ is divisible by $p$, then it is apparent that
$a^p - a$ is divisible by $p$. Assume $p \nmid a$.
\begin{enumerate}
\item
Consider the set 
\[
	\{\ol{1}, \ol{2}, \dots, \ol{p - 1}\}
\]
of nonzero congruence classes modulo $p$.
Then consider the set
\[
	\{\ol{a}, \ol{2a}, \dots, \ol{(p - 1)a}\}
\]
of congruence classes modulo $p$.

\item
We claim that they're the same set. Indeed, since
both sets have $p - 1$ elements, we just need to
show that $\ol{j}$ appears in the second set
for every $j \in \{1, \dots, p - 1\}$.

In other words, we want $ka \equiv j \pmod p$ for some 
$k \not\equiv 0 \pmod p$.
Let $b$ be such that $ab \equiv 1 \pmod p$, and
let $k = jb$. Then
\[
	ka \equiv jba \equiv j \pmod p.
\]
Obviously $k \not\equiv 0 \pmod p$ since $j \not\equiv 0
\pmod p$.

\item
Then
\begin{align*}
	1 \cdot 2 \cdots (p - 1) &\equiv
	a \cdot (2a) \cdots (p - 1)a \\
	&\equiv 1 \cdot 2 \cdots (p-1) \cdot a^{p-1} \pmod p.
\end{align*}
Multiplying both sides by an inverse of $(p-1)!$
gives 
\[
	a^{p-1} \equiv 1 \pmod p.
\]
\end{enumerate}
\end{proof}

\subsection{$(\ZZ/n\ZZ)^\times$}

Corollaries \ref{prime} and \ref{inverse} are not true
if $p$ is not prime. 
For example, $4 | 2 \cdot 2$
but $4$ does not divide $2$, and there is no integer
$b$ such that $2b \equiv 1 \pmod 4$, because $2b$ cannot
be odd.

Here are some generalizations of them to general $n$.

\begin{lemma}
Suppose $n$ be a positive integer. If $n | ab$, then $b$
is a multiple of $n/\gcd(a, n)$.
\end{lemma}
\begin{proof}
Let $d = \gcd(a, n)$. Suppose 
\[
	d = ra + sn.
\]
Then $db = rab + snb$ is a multiple of $n$, so
$b$ is a multiple of $n/d$.
\end{proof}

\begin{lemma}
Let $n$ be a positive integer, and $a$ be an integer 
such that $\gcd(a, n) = 1$.
There exists an integer $b$ such that
$ab \equiv 1 \pmod n$.
\end{lemma}
\begin{proof}
	Since $\gcd(a, n) = 1$, there exist $r, s \in \ZZ$
	such that
	\[
		1 = ra + sn.
	\]
	So $ra \equiv 1 \pmod p$, and we can take $b = r$.
\end{proof}

\begin{definition}
	Let $(\ZZ/n\ZZ)^\times$ denote the set of congruence
	classes $\ol a$ modulo $n$ such that $\gcd(a, n) = 1$.
	Note that this does not depend on the choice of $a$,
	only on $a \pmod n$, since $\gcd(a + nk, n) = \gcd(a, n)$
	as mentioned previously.
\end{definition}

\begin{definition}
In the special case when $n = p$ is a prime,
$(\ZZ/p\ZZ)^\times$ is just all of the elements of
$\ZZ/p\ZZ$ other than $\ol 0$.
\end{definition}

\begin{definition}
Let $a$ and $b$ be integers, both not zero. The
\emph{least common multiple} of $a$ and $b$,
denoted $\lcm(a, b)$ is the smallest positive integer
which is a multiple of both $a$ and $b$.
\end{definition}

\begin{proposition}
Let $a$ and $b$ be positive integers.
If $d = \gcd(a, b)$ and $m = \lcm(a, b)$, then
$ab = dm$.	
\end{proposition}
\begin{proof}
	Suppose $m = ak$. Since $b | m$, by the lemma,
	$k \ge b/d$, so $m \ge ab/d$.
	On the other hand, it is clear that $ab/d$
	is a multiple of both $a$ and $b$, so
	$m \le ab/d$.
\end{proof}

\end{document}
