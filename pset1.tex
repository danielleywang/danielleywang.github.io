\documentclass[11pt]{article}

\usepackage{amssymb}
\usepackage{xcolor}
\usepackage{todonotes}
\usepackage[margin=1.25in]{geometry}

\newcommand{\ZZ}{\mathbb Z}

\usepackage{amsthm}
\theoremstyle{definition}
\newtheorem{problem}{Problem}
\newtheorem*{bonus}{Bonus problem (not graded)}

\usepackage[inline,shortlabels]{enumitem}

\begin{document}

\section*{Math 401 Problem Set 1 (due January 16, 2026)}
When you upload your work to Gradescope, please
select the pages corresponding to each problem.

At the top of your submission, write ``sources
consulted:'' and list all sources you used while working
on this problem set. If none, write ``none.''

\bigskip

\begin{problem}
  Let $a$, $b$, $c$ be integers, not all zero.
  We define $\gcd(a, b, c)$ to be the largest integer
  which divides all of $a$, $b$, $c$.

  Prove that if $\gcd(a,b,c)=1$,
  then there exist integers $x$, $y$, $z$ such that $ax+by+cz=1$.
\end{problem}

\begin{problem}
  Let $G$ be a group.
  In this problem we'll fill in some of the proofs omitted from the lecture.
  \begin{enumerate}[(a)]
    \item The group axioms guarantee that for every element $a$,
    there is an inverse $b$ satisfying $ab = ba = 1$.
    Show that the inverse promised by this axiom is \emph{unique}:
    that if $ab = ba = 1$ and $ab' = b'a = 1$, then $b = b'$.
    This justifies the notation ``$a^{-1}$'' and referring to it as ``the'' inverse.

    \item Prove that $(ab)^{-1} = b^{-1} a^{-1}$.
  \end{enumerate}
\end{problem}

\begin{problem}
  For each of the following sets and operations,
  determine whether they form a valid group.
  For yes, briefly explain why each axiom is true (in some cases ``it's obvious'' is acceptable). For no, specify which axiom fails and briefly explain why.
  \begin{enumerate}[(a)]
  \item The set of positive real numbers, under multiplication.
  \item The set of $2 \times 2$ real symmetric matrices
  with nonzero determinant, under multiplication.
  \item The set of $2 \times 2$ matrices with integer entries and
  nonzero determinant, under multiplication.
  \item The set of $2 \times 2$ matrices with integer entries and
  determinant equal to $1$, under multiplication.
  \item The set of odd decimal digits $\{1,3,5,7,9\}$,
  where $a \ast b$ is the last decimal digit of $ab$.
  \end{enumerate}
\end{problem}

\begin{problem}
  How many elements of order $2$ are there in each of these groups?
  (Note that $1013$ is prime.)

  \begin{enumerate*}[(a)]
    \item $\ZZ/24\ZZ$.
    \item $\ZZ/25\ZZ$.
    \item $(\ZZ/1013\ZZ)^\times$.
  \end{enumerate*}
\end{problem}

\begin{problem}
  Let $G = S_7$, $g = (1 \; 2)(3 \; 4)$
  and $h = (1 \; 2 \; 3 \; 4 \; 5 \; 6 \; 7)$.
  Compute the orders of $gh$ and $hg$.
\end{problem}

\begin{problem}
  How many subgroups does $\ZZ/7\ZZ \times D_2$?
  (Note: $D_2$ is the dihedral group of order $4$.)
\end{problem}

\begin{bonus}
  Let $p$ be an odd prime. Prove that $p$ divides $(p-1)!+1$.
\end{bonus}

\begin{bonus}
  Let $n > 1$ be a positive integer. Prove that $n$ does not divide $2^n - 1$.
\end{bonus}

\end{document}
